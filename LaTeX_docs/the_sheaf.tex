\documentclass{amsart}
\usepackage{amsmath}
\usepackage{amsthm}
%\usepackage{a4wide}
%\usepackage{enumerate}

\newcommand{\R}{\mathbb{R}}
\newcommand{\Z}{\mathbb{Z}}
\DeclareMathOperator{\Spa}{Spa}
\DeclareMathOperator{\Spv}{Spv}
\DeclareMathOperator{\pre}{pre}

\theoremstyle{plain}
\newtheorem{theorem}{Theorem}
\newtheorem{lemma}[theorem]{Lemma}
\newtheorem{corollary}[theorem]{Corollary}
\newtheorem{proposition}[theorem]{Proposition}
\theoremstyle{remark}
\newtheorem{remark}[theorem]{Remark}
\newtheorem*{remarkn}{Remark}

\begin{document}

\title{Adic spaces -- more details of what remains}

\section{The category}

An adic space is a topological space with some extra structure, with the property that the space has a cover by open sets $U$, the structure pulls back to these sets $U$, and each $U$ is isomorphic to $\Spa(A)$ for some Huber pair $A$. The isomorphism is a structure-preserving isomorphism.

The notion of isomorphism does not change if you replace a category with a full subcategory, or embed your category fully into a bigger category. In short, we are flexible with what this extra structure is. In particular, removing or adding extra structure which does not change the morphisms will not affect anything. For example, it doesn't matter if the structure is a sheaf of complete topological rings or a presheaf of topological rings, because a morphism of spaces equipped with a sheaf of complete topological rings is by definition a morphism of the underlying spaces equipped with a presheaf of topological rings.

The less structure we have, the easier our life is, if we are merely concerned with writing down the \emph{definition} of an adic space (which we need for the definition of a perfectoid space, because a perfectoid space is an adic space with some more structure). The reason our life gets easier is that the less structure we have, the less structure we need to put on $\Spa(A)$, which is where the remaining work in the project is.

So I propose that we work in the following category: an object is a topological space, a presheaf of topological rings on that topological space (which is in particular a presheaf of rings), and an equivalence class of valuation on each stalk (the stalks of the presheaf of rings, no topology involved on the rings at all in the definition of stalk). We could even work in the larger category where we just demand a random preorder, or even a random binary relation, on the stalks.

These categories I am suggesting have no canonical name that I know of.

We will need to prove that if $(X,\mathcal{F},v_x)$ is an object of this category then given an open subset of $X$ the structure pulls back. I don't envisage any problems with this. We will need some fact about stalks not changing, which will come from the universal property.

Our definition will hence be easy, once we have managed to give $\Spa(A)$ the structure of an object of this category.

\section{The presheaf.}

If $A=(R,R^+)$ is a Huber pair then $\Spa(A)$ is the space of equiv classes of continuous valuations which are bounded by 1 on $R^+$. We have all this. That is the only role that $R^+$ plays and we can now forget about it. We don't need the sheaf $\mathcal{O}^+$ for the definition of a perfectoid space, as far as I can remember. We certainly don't need it for the definition of an adic space.

So now let's focus on $R$. Say a pair $(T,s)$ consisting of a subset of $R$ and an element of $r$ is \emph{nice} if the subset is finite and the ideal generated by the subset is open. I don't know of a good name for these subsets but they play a key role.

Attached to nice data $(T,s)$ is an open set $D(T,s)\subseteq\Spa(A)$ -- a ``rational open subset''. The naive idea is to attach the ring $R\langle T/s\rangle$ to the open set $D(T,s)$, but there's a problem here: distinct pieces of nice data can give rise to the same open subsets, and in contrast to the schemes project we do not have the theorem that when this happens the associated rings are canonically isomorphic (by which I mean there's a unique isomorphism with some properties). 

Here's the way that we ensure that if $D(T_1,s_1)=D(T_2,s_2)$ then we can identify the elements of $R\langle T_1/s_1\rangle$ and $R\langle T_2/s_2\rangle$. WLOG $s_i\in T_i$. What we \emph{can} do is push them both forward to the isomorphic ring $R\langle T_1T_2/s_1s_2\rangle$ and demand that the images are equal. The maps $R\langle T_i/s_i\rangle\to R\langle T_1T_2/s_1s_2\rangle$ exist by the universal property (which I'll review now).

More generally, recall that we know that $R(T/s)$ is universal for continuous maps $f:R\to B$ with $B$ a nonarch ring, $f(s)$ invertible and $f(T)/f(s)$ power-bounded. This is Wedhorn 5.51. So we could define $(T_1,s_1)\leq(T_2,s_2)$ iff $s_1$ is in the units of $R(T_2/s_2)$ and $T_1/s_1\subseteq R(T_2/s_2)$ is power-bounded. If $(T_1,s_1)\leq(T_2,s_2)$ then there's a map $R(T_1/s_1)\to R(T_2/s_2)$ and hence a map $R\langle T_1/s_1\rangle\to R\langle T_2/s_2\rangle$. Now if $U\subseteq\Spa(A)$ is an arbitrary open, then we can define $\mathcal{O}_X(U)$ to be the subset of $\prod_{(T,s):D(T,s)\subseteq U}R\langle T/s\rangle$ consisting of the elements such that if $D(T_i,s_i)\subseteq U$ and $(T_1,s_1)\leq (T_2,s_2)$, then the induced map $R\langle T_1/s_1\rangle \to R\langle T_2/s_2\rangle$ identifies one element with the other.

This is easily checked to be a ring; if we give it the subspace topology of the product topology then it's a topological ring. It's also easily checked to be a presheaf. In maths we can also prove $(T_1,s_1)\leq(T_2,s_2)\implies D(T_1,s_1)\subseteq D(T_2,s_2)$ and conversely if $D(T_1,s_1)\subseteq D(T_2,s_2)$ and $s_i\in T_i$ then $(T_1,s_1)\leq (T_1T_2,s_1s_2)$, $(T_2,s_2)\leq (T_1T_2,s_1s_2)$, and the map $R\langle T_2/s_2\rangle \to R\langle T_1T_2/s_1s_2\rangle$ is an isomorphism (although we can't prove it), so we're doing the right thing -- just in a slightly weird way.

\section{The valuations.}

What we need now is the (equivalence class of) valuation on the stalks. If $v\in\Spa(A)$ then $v$ gives us a valuation on $R$. If we can (a) extend $v$ to $v_{(T,s)}$ on $R\langle T/s\rangle$ for every $(T,s)$ with $v\in D(T,s)$ and (b) show that if $v\in D(T_1,s_1)\cap D(T_2,s_2)$ and $(T_1,s_1)\leq (T_2,s_2)$ then the pullback of $v_{(T_2,s_2)}$ under the map $R\langle T_1/s_1\rangle\to R\langle T_2/s_2\rangle$ is $v_{(T_1,s_1)}$, then we're home -- the rest is formal. 

\end{document}
